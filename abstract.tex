\chapter*{Abstract}
Many problems in computer vision lead to polynomial systems solving.
The state of the art algebraic methods for polynomial systems solving are able to efficiently solve the systems over complex numbers.
In computer vision and robotics non-real solutions are then discarded, as they are not solutions of the original geometric problems.
On this purpose, we review and implement the moment method for polynomial systems solving, which solves the problems over real numbers directly.
We show that the moment method is applicable to the minimal problems from geometry of computer vision.
For that, we give description of the calibrated camera pose problem and of the calibrated camera pose with unknown focal length problem.
We compare our implementation of the moment with the state of the art methods on these two selected minimal problems on real 3D scenes.

Moreover, we review and implement a method for solving polynomial optimization problems, which can extend the moment method with inequality constraints.
This method uses Lasserre's hierarchies to find the optimal values of the original optimization problems.
We compare the performance of our implementation with the state of the art methods on synthetically generated polynomial optimization problems.

Since the semidefinite programs solving is a key element in the moment method and the polynomial optimization methods, we review and implement an interior-point algorithm for semidefinite programs solving.
We compare the performance of our implementation with the state of the art methods on synthetically generated semidefinite programs.

\paragraph{Keywords:}
computer vision, polynomial systems solving, polynomial optimization, semidefinite programming, minimal problems

\begin{otherlanguage}{czech}
\chapter*{Abstrakt}
Mnoho problémů v počítačovém vidění vede na řešení systémů polynomiálních rovnic.
Současné metody na řešení systémů polynomiálních rovnic jsou schopny řešit tyto systémy v oboru komplexních čísel.
V počítačovém vidění a robotice jsou nereálná řešení následně vyřazena, protože ta nejsou řešeními původních geometrických problémů.
Z~tohoto důvodu prozkoumáme a implementujeme metodu momentů pro řešení systémů polynomiálních rovnic, která řeší tyto problémy přímo v oboru reálných čísel.
Ukážeme, že metoda momentů je použitelná na minimální problémy z geometrie počítačového vidění.
Proto popíšeme problém nalezení polohy kalibrované kamery a problém nalezení polohy kalibrované kamery s neznámou ohniskovou vzdáleností.
Na těchto dvou vybraných minimálních problémech a reálných 3D scénách porovnáme naší implementaci metody momentů se sou\-čas\-ný\-mi metodami.

Dále prozkoumáme a implementujeme metodu na řešení polynomiálně optimalizačních problémů, která může rozšířit metodu momentů o omezení s nerovnostmi.
Tato metoda využívá Lasserrových hierarchií k nalezení optimálních hodnot původních optimalizačních problémů.
Na synteticky generovaných polynomiálně optimalizačních prob\-lé\-mech porovnáme výkon naší implementace se současnými metodami.

Protože řešení semidefinitních programů je klíčovým elementem metody momentů a metod polynomiální optimalizace, prozkoumáme a implementujeme algoritmus vnitřních bodů na řešení semidefinitních programů.
Na synteticky generovaných semidefinitních problémech porovnáme výkon naší implementace se současnými metodami.

\paragraph{Klíčová slova:}
počítačové vidění, řešení polynomiálních systémů, polynomiální optimalizace, semidefinitní programování, minimální problémy
\end{otherlanguage}
