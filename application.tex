\chapter{Minimal problems in computer vision geometry}\labelcha{app}
Many problems from computer vision geometry can be modeled by systems of polynomial equations.
A problem that requires only the minimal subset of data points to solve the problem is called a minimal problem.
A typical example is the 5-point algorithm \cite{5pt} for relative pose estimation between two cameras given five image correspondences only.
In many applications, solvers of these minimal problems are used in the Random Sample Consensus (RANSAC) algorithm \cite{ransac}, where the minimal problems has to solved repeatedly for a large amount of input data.
Thus, these solvers are required to be fast and efficient.
The state of the art method is to generate these solvers by automatic generators \cite{autogen}, which are based on Gr\"obner basis construction and eigenvectos of multiplication matrices computation.
In these solvers both real and non-real are computed, but the non-real solutions are discarded, since they have no geometric meaning.

In \refsec{POP:sol}, we have proposed and implemented an algorithm, which does not need to compute the superfluous non-real solutions, and therefore may be faster than the standard solvers generated by the automatic generators.
In this section, we compare the speed and numerical stability of the state of the art solvers with our implementation of the moment method algorithm for polynomial system solving.
For this reason, we have selected few minimal problems from computer vision geometry, on which we will compare the selected solvers.
