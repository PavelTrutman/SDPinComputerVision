\chapter{Optimization over polynomials}\labelcha{POP}

\section{State of the art review}

\section{Algebraic preliminaries}
In this whole chapter focused on polynomial optimization and polynomial systems solving, we will follow the notation from \cite{Cox-Little-Shea97}.
Just to keep this chapter self-contained, we will recall some basics of polynomial algebra.

\subsection{The polynomial ring, ideals and varieties}
Firstly, the ring of multivariate polynomials in $n$ variables with coefficients in $\R$ is denoted as $\R[x]$, where $x = \bmB x_1 & x_2 & \cdots & x_n \bmE^\top$.
For $\alpha_1, \alpha_2, \ldots, \alpha_n\in\N$, $x^\alpha$ denotes the monomial ${x_1}^{\alpha_1}\cdot{x_2}^{\alpha_2}\cdots{x_n}^{\alpha_n}$, with total degree $|\alpha| = \sum_{i=1}^n \alpha_i$, where $\alpha = \bmB \alpha_1 & \alpha_2 & \cdots & \alpha_n\bmE^\top$.
A polynomial $p\in\R[x]$ can be written as
\begin{eqnarray}
  p = \sum_{\alpha\in\N^n} p_\alpha x^\alpha
\end{eqnarray}
with total degree $\deg(p) = \max_{\alpha\in\N^n}|\alpha|$ and coefficients $p_\alpha\in\R$.

A linear subspace $I \subseteq \R[x]$ is an ideal if $p\in I$ and $q\in\R[x]$ implies $pq \in I$.
Let $f_1, f_2, \ldots, f_m$ be polynomials in $\R[x]$. Then the set
\begin{eqnarray}
  \langle f_1, f_2, \ldots, f_m\rangle &=& \Bigg\{\sum_{j=1}^mh_jf_j\ |\ h_1, h_2, \ldots, h_m\in\R[x]\Bigg\}
\end{eqnarray}
is called the ideal generated by $f_1, f_2, \ldots, f_m$.
Given the ideal $I\in\R[x]$, the algebraic variety of $I$ is the set
\begin{eqnarray}
  V_\C(I) &=& \big\{x\in\C^n\ |\ f(x) = 0 \text{ for all } f\in I\big\}
\end{eqnarray}
and its real variety is
\begin{eqnarray}
  V_\R(I) &=& V_\C(I) \cap \R^n.
\end{eqnarray}
The ideal $I$ is said to be zero-dimensional when its complex variety $V_\C(I)$ is finite.
The vanishing ideal of a subset $V\subseteq\C^n$ is the ideal
\begin{eqnarray}
  \Ideal(V) &=& \big\{f\in\R[x]\ |\ f(x) = 0 \text{ for all } x\in V\big\}.
\end{eqnarray}
The radical ideal of the ideal $I\subseteq \R[x]$ is the ideal
\begin{eqnarray}
  \sqrt{I} &=& \big\{f\in\R[x]\ |\ f^m\in I \text{ for some } m\in\Z^+\big\}.
\end{eqnarray}
The real radical ideal of the ideal $I\subseteq \R[x]$ is the ideal
\begin{eqnarray}
  \sqrt[\R]{I} &=& \big\{f\in\R[x]\ |\ f^{2m} + \sum_j h_j^2 \in I \text{ for some } h_j\in\R[x], m\in\Z^+\big\}.
\end{eqnarray}

The following two theorems are stating the relations between the vanishing and (real) radical ideals.

\begin{theorem}[Hilbert's Nullstellensatz]
  Let $I\in\R[x]$ be an ideal. The radical ideal of $I$ is equal to the vanishing ideal of its variety, i.e.\
  \begin{eqnarray}
    \sqrt{I} &=& \Ideal\big(V_\C(I)\big).
  \end{eqnarray}
\end{theorem}

\begin{theorem}[Real Nullstellensatz]
  Let $I\in\R[x]$ be an ideal. The real radical ideal of $I$ is equal to the vanishing ideal of its real variety, i.e.\
  \begin{eqnarray}
    \sqrt[\R]{I} &=& \Ideal\big(V_\R(I)\big).
  \end{eqnarray}
\end{theorem}

The quotient ring $\R[x]/I$ is the set of all equivalence classes of polynomials in $\R[x]$ for congruence modulo ideal $I$
\begin{eqnarray}
  \R[x]/I &=& \big\{[f]\ |\ f\in\R[x]\big\},
\end{eqnarray}
where the equivalence class $[f]$ is
\begin{eqnarray}
  [f] &=& \big\{f+g\ |\ q\in I\big\}.
\end{eqnarray}
Because $\R[x]/I$ is a ring, it is equipped with addition and multiplication on the equivalence classes:
\begin{eqnarray}
  [f] + [g] &=& [f + g]\\
  ~[f][g] &=& [fg]
\end{eqnarray}
for $f, g\in\R[x]$.

For an ideal $I$, there is a relation between the dimension of $\R[x]/I$ and the cardinality of the variety $V_\C(I)$: if $V_\C(I)$ a finite set, then
\begin{eqnarray}
  |V_\C(I)| &\leq& \dim\big(\R[x]/I\big).
\end{eqnarray}
Moreover, if $I$ is a radical ideal, then
\begin{eqnarray}
  |V_\C(I)| &=& \dim\big(\R[x]/I\big).
\end{eqnarray}

Assume that the number of complex roots is finite and let $N = \dim\big(\R[x]/I\big)$, and therefore $|V_\C(I)|\leq N$.
Consider a set $\Base = \{b_1, b_2, \ldots, b_N\} \subseteq \R[x]$ for which the equivalence classes $[b_1], [b_2], \ldots, [b_N]$ are pairwise distinct and $\big\{[b_1], [b_2], \ldots, [b_N]\big\}$ is a basis of $\R[x]/I$.
Then every polynomial $f\in\R[x]$ can be written in unique way as
\begin{eqnarray}
  f &=& \sum_{i=1}^Nc_ib_i + p,
\end{eqnarray}
where $c_i\in\R$ and $p\in I$.

\subsection{Companion matrices}

\section{Moment matrices}

\section{Polynomial optimization}

\section{Solving systems of polynomial equations}
