\chapter{Semidefinite programming}
%TODO Some text about semidefinite programming ...

\section{Preliminaries on semidefinite programs}
We introduce here some notation and preliminaries about symmetric matrices and semidefinite programs.
We will introduce further notation and preliminaries later on in the text when needed.

At the beginning, let us denote the inner product for two vectors $x$, $y \in\R^n$
\begin{eqnarray}
  \langle x, y\rangle &=& \sum_{i=1}^n x_iy_i
\end{eqnarray}
and the Frobenius inner product for two matrices $X$, $Y\in\R^{n\times m}$.
\begin{eqnarray}
  \langle X, Y\rangle &=& \sum_{i=1}^n \sum_{j=1}^m X_{ij}Y_{ij}
\end{eqnarray}

\subsection{Symmetric matrices}
Let $\Sym_n$ denotes the space of $n\times n$ real symmetric matrices.

For matrix $M\in\Sym_n$, the notation $M \succeq 0$ means that $M$ is positive semidefinite.
$M \succeq 0$ if and only if any of the following equivalent properties holds.
\begin{enumerate}
  \item $x^\top Mx \geq 0$ for all $x \in \R^n$.
  \item All eigenvalues of $M$ are nonnegative.
\end{enumerate}

For matrix $M\in\Sym_n$, the notation $M \succ 0$ means that $M$ is positive definite.
$M \succ 0$ if and only if any of the following equivalent properties holds.
\begin{enumerate}
  \item $M \succeq 0$ and $\rank M = n$.
  \item $x^\top Mx > 0$ for all $x \in \R^n$.
  \item All eigenvalues of $M$ are positive.
\end{enumerate}

\subsection{Semidefinite programs}
The standard (primal) form of a semidefinite program in variable $X\in\Sym_n$ is defined as follows:
\begin{eqnarray}
  \begin{array}{rclrrclcll}
    p^* &=& \displaystyle \sup_{X\in\Sym_n} & \multicolumn{3}{l}{\langle C,X\rangle} \\
    && \text{s.t.} & \langle A_i, X\rangle &=& b_i & (i = 1,\ldots,m)\\
    &&& X &\succeq& 0
  \end{array}
\end{eqnarray}
where $C$, $A_1$, \ldots, $A_m \in \Sym_n$ and $b\in\R^m$ are given.

The dual form of the primal form is the following program in variable $y\in\R^m$.
\begin{eqnarray}
  \begin{array}{rclrrclcll}
    d^* &=& \displaystyle \inf_{y\in\R^m} & \multicolumn{3}{l}{b^\top y} \\
    && \text{s.t.} & \displaystyle \sum_{i=1}^m A_iy_i - C &\succeq& 0
  \end{array}
\end{eqnarray}

\section{State of the art review}

\section{Theoretical background}

\section{Nesterov's approach}
In this section, we will follow Chapter 4 of \cite{Nesterov-2004} by Y. Nesterov, which is devoted to convex minimization problems.
We will extract from it only the minimum, just to be able to introduce a algorithm for SDP programs solving.
We will present some basic definitions and theorems, but we will not prove them.
For the proofs look into \cite{Nesterov-2004}.

\subsection{Self-concordant functions}
\begin{definition}[Self-concordant function in $\R$]
  A closed convex function $f: \R \mapsto \R$ is self-concordant if there exist a constant $M_f \geq 0$ such that the inequality
  \begin{eqnarray}
    |f'''(x)| &\leq& M_f f''(x)^{3/2}
  \end{eqnarray}
  holds for all $x\in\dom f$.
\end{definition}

For better understanding of self-concordant functions we provide several examples.

\begin{example}~\\[-0.5cm]
  \begin{enumerate}
    \item Linear and convex quadratic functions.
      \begin{eqnarray}
        f'''(x) &=& 0\ \text{ for all } x
      \end{eqnarray}
      Linear and convex quadratic functions are self-concordant with constant $M_f = 0$.
    \item Negative logarithms.
      \begin{eqnarray}
        f(x) &=& -\ln(x)\ \text{ for } x>0\\
        f'(x) &=& -\frac{1}{x}\\
        f''(x) &=& \frac{1}{x^2}\\
        f'''(x) &=& -\frac{2}{x^3}\\
        \frac{|f'''(x)|}{f''(x)^{3/2}} &=& 2
      \end{eqnarray}
      Negative logarithms are self-concordant functions with constant $M_f = 2$.

    \item Exponential functions.
      \begin{eqnarray}
        f(x) &=& e^x\\
        f''(x) \ =\ f'''(x) &=& e^x\\
        \frac{|f'''(x)|}{f''(x)^{3/2}} &=& e^{-x/2} \rightarrow\infty \ \text{ as } x\rightarrow-\infty
      \end{eqnarray}
      Exponential functions are not self-concordant functions.
  \end{enumerate}
\end{example}

\begin{definition}[Self-concordant function in $\R^n$]
  A closed convex function $f: \R^n \mapsto \R$ is self-concordant if function
  \begin{eqnarray}
    g(t) &=& f(x + tv)
  \end{eqnarray}
  is self-concordant for all $x\in\dom f$ and all $v\in\R^n$.
\end{definition}

Now, let us focus on the main properties of self-concordant functions.

\begin{theorem}
  Let functions $f_i$ be self-concordant with constants $M_i$  and let $\alpha_i > 0$, $i = 1,2$. Then the function
 \begin{eqnarray}
   f(x) &=& \alpha_1f_1(x) + \alpha_2f_2(x)
 \end{eqnarray}
 is self-concordant with constant
 \begin{eqnarray}
   M_f &=& \max \bigg\{\frac{1}{\sqrt{\alpha_1}}M_1, \frac{1}{\sqrt{\alpha_2}}M_2\bigg\}
 \end{eqnarray}
 and
 \begin{eqnarray}
   \dom f &=& \dom f_1 \cap \dom f_2.
 \end{eqnarray}
\end{theorem}

\begin{corollary}
  Let function $f$ be self-concordant with some constant $M_f$ and $\alpha > 0$. Then the function $\phi(x) = \alpha f(x)$ is also self-concordant with the constant $M_\phi = \frac{1}{\sqrt{\alpha}}M_f$.
\end{corollary}

\begin{theorem}
  Let function $f$ be self-concordant. If $\dom f$ contains no straight line, then the Hessian $f''(x)$ is nondegenerate at any $x$ from $\dom f$.
\end{theorem}

\section{Implementation details}

\section{Comparison with the state of the art methods}

